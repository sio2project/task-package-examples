\documentclass[en]{spiral}
  \def\title{Tree}
  \def\id{tre}
  \def\contest{Example task}
  \def\desc{}
  \def\TL{1 s}
  \def\ML{256 MB}

\begin{document}
  \makeheader

  As we all know, every Christmas tree consists of exactly three parts,
  in their descending order: upper part, lower part, and the trunk.
  Upper and lower part of tree are represented by an isosceles triangle 1 character wide
  at the top and growing by 1 character in both directions (left and right) 
  with each line, going downwards.
  
  A Christmas tree will be said to be of size $n$ if its upper part
  is $n$ lines tall and its lower part is $n + 1$ lines tall. 
  The trunk is always 2 characters tall and 1 character wide, located on
  the tree's vertical axis, regardless of the overall tree size.
  In this problem you'll be asked to print a Christmas tree of given size,
  represented by hashes on a dotted background.

  \section{Input}

    A single integer $n$ ($n \leq 1\,000$) denoting the size of the tree.

  \section{Output}

    If $n=0$, print only one line saying ``Too small to exist''.\\
    Otherwise, print $2 \cdot n + 3$ lines, $2 \cdot n + 1$ characters each, representing a drawing 
    of an $n$–sized Christmas tree, as described in the problem statement.
    Check out the examples in order to get a better understanding of the output format.

  \example{1ocen}

\end{document}
