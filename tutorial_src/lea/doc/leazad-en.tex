\documentclass{spiral}
  \def\title{Leaves}
  \def\id{lea}
  \def\contest{Example task}
  \def\desc{}
  \def\TL{1 s}
  \def\ML{64 MB}

\begin{document}
    \makeheader

    \noindent A tree is a connected, undirected and unweighted graph,
    consisting of $n$ vertices and $n-1$ edges.
    The characteristic feature of the tree is that it is connected and acyclic.
    A leaf in a tree is a vertex that has its degree equal to 1
    (that is, only one edge is incident to it).
    If a rooted tree is given, it means that a certain designated vertex is a root.
    It is assumed that in a rooted tree the root is not called a leaf
    (even if its degree is equal to 1).
    You are given a $n$ - vertex rooted tree.
    The root is the vertex with index 1.
    Your task is to calculate the number of it's leaves.

\section{Input}

    In the first line of input one number $n$ ($n \leq 2 \cdot 10^5$) is given,
    denoting the number of vertices of the tree.
    The next $n-1$ lines contain information about the edges in the tree.
    There are two different numbers $u_i$ and $v_i$ ($1 \leq u_i, v_i \leq n$)
    given on the $i$-th line, denoting vertices connected by the $i$-th edge.

\section{Output}

    Output one number - number of leaves.

    \example{0}

\section{Scoring}

    \begin{center}
        \begin{tabular}{|c|p{5cm}|c|}
            \hline
            \textbf{Subtask} & \textbf{Constraints} & \textbf{Points} \\
            \hline
            1 & $n \leq 100$ & 30 \\
            \hline
            2 & There are at least 2 edges incident to the root & 20 \\
            \hline
            3 & no additional constraints & 50 \\
            \hline
        \end{tabular}
    \end{center}

\end{document}
