\documentclass{spiral}
  \def\title{Leaves}
  \def\id{lea}
  \def\contest{Example task}
  \def\desc{}
  \def\TL{1 s}
  \def\ML{64 MB}

\begin{document}
    \makeheader

    Mamy $n$  - wierzchołkowe drzewo ukorzenione. Powiedz, ile jest liści.
    PS: Korzeniem jest wierzchołek 1 i korzeń z definicji nie jest liściem.

\section{Input}

    A single integer $n$ ($n \leq 2 \cdot 10^5$) denoting the size of the tree.
    Next $n-1$ rows contain two numbers stating edges of the tree.
    $u \neq v$, edges are not repeating.

\section{Output}

    Output one number - number of leaves.

    \example{0}

\section{Scoring}

    Podzadanie 1: n <= 100
    Podzadanie 2: stopień korzenia jest > 1
    Podzadanie 3: brak dodatkowych założeń

\end{document}
