\documentclass{spiral}
  \def\title{Guess the number}
  \def\id{gue}
  \def\contest{Example task}
  \def\desc{}
  \def\TL{1 s}
  \def\ML{64 MB}

\begin{document}
  \makeheader

  Dana jest dwuosobowa gra "zgadnij liczbę". Zasady są następujące:
  Pierwszy gracz wybiera liczbę z przedziału [1, $n$],
  oraz podaje drugiemu graczowi $n$ ($1 \leq n \leq 10^6$).
  Drugi gracz musi zgadnąć wybraną liczbę. 
  Do dyspozycji ma zapytania postaci "czy szukana liczba jest większa niż $x$?"
  Napisz program, który pomoże drugiemu graczowi zgadnąć liczbę tak,
  by nie musiał użyć zbyt welu zapytań.

  \section{Biblioteka}
    Jest to zadanie interaktywne,
    to znaczy Twój program będzie porozumiewał się z biblioteką.
    Aby użyć biblioteki, należy załączyć nagłówek \texttt{\#include "{}gue.h"{}}.
    Biblioteka udostępnia następujące funkcje:
    \begin{itemize}
        \item \texttt{int init()} -- funkcja ta powinna zostać wywoływana
        tylko raz, na początku działania programu.
        Rozpoczyna ona grę, a gracz pierwszy wymyśla szukaną liczbę.
        Zwraca liczbę $n$, podaną przez tego gracza.

        \item \texttt{bool isGreater(int x)} -- funkcja ta pyta pierwszego gracza,
        czy wybrana przez niego liczba jest większa od $x$.
        Zwraca \texttt{true} jeśli jest, \texttt{false} w przeciwnym wypadku.
        Podana liczba powinna należeć do przedziału [1, $n$].

        \item \texttt{void answer(int x)} -- funkcja ta zgaduje,
        jaka jest liczba pierwszego gracza. To kończy grę.
        Powinna być ona wywołana tylko raz,
        a po jej wykonaniu program powinien się zakończyć.
    \end{itemize}

  \section{Kompilacja na swoim komputerze}
    Pliki z archiwum \texttt{gue\_dla\_zaw.zip} dostepnego w zakładce "Pliki"
    należy wypakować do folderu z kodem źródłowym programu.
    Aby program się skompilował należy załączyć nagłówek
    \texttt{\#include "gue.h"}.\\
    Program należy skompilować razem z biblioteką \texttt{gue\_lib.cc}.
    Można to zrobić za pomocą polecnia: \\
    \texttt{g++~twoj\_program.cpp~gue\_lib.cc~-o~twoj\_program}.
    Warto mieć na uwadze, że przykładowa biblioka nie sprawdza
    poprawności wywoływanych komend.

  \section{Wyjście}
    Twój program nie powinien pisać na standardowe wyjście (\texttt{stdout})
    ani czytać ze standardowego wejścia (\texttt{stdin}).
    Dozwolone jest pisanie na standardowe wyjście diagnostyczne (\texttt{stderr}),
    lecz pamiętaj, że zabiera to cenny czas.

  \section{Przykład}
    \begin{tabular}{|c|c|p{12.5cm}|}
      \hline

      \textbf{Funkcja} & \textbf{Wynik} & \textbf{Opis} \\ \hline

      \texttt{init()} & 5 &
      Gra się rozpoczyna. Gracz pierwszy wybiera liczbę 4,
      a szukamy liczby w przedziale [1, 5], o czym jesteśmy poinformowani
      wartością funkcji init().
      \\ \hline

      \texttt{isGreater(4)} & false &
      Pytamy czy szukana liczba jest większa od 4.
      Nie jest ona większa (gdyż jest ona równa),
      więc otrzymujemy odpowiedź \texttt{false}. 
      \\ \hline

      \texttt{isGreater(3)} & true &
      Pytamy czy szukana liczba jest większa od 3.
      Jest ona większa, więc otrzymujemy odpowiedź \texttt{true}. 
      \\ \hline

      \texttt{answer(4)} & - &
      Wiemy już, że szukaną liczbą jest 4,
      więc udzielamy odpowiedzi i kończymy program.
      \\ \hline
    \end{tabular}

\section{Ocenianie}
    Żeby program dostał jakiekolwiek punkty, musi on działać zgodnie z wymaganiami, czyli jako pierwszą wywołać funkcję \texttt{init}, jako ostatnią \texttt{answer} i każdą z nich dokładnie raz.
    Musi on również stosować się do zasad opisanych w sekcji wyjście.
    Jeżeli naruszy on którąś z tych zasad, otrzyma werdykt WA.\\
    Jeśli powyższe warunki zostaną spełnione,
    to program oceniany jest w następujący sposób:\\
    Jeżeli zgadnięta liczba jest niepoprawna, nie otrzyma on punktów za dany test.\\
    W przeciwnym wypadku niech $k$ oznacza liczbę zapytań.
    \begin{itemize}
    \item Jeśli $k$ nie przekracza 20, program dostanie pełną liczbę punktów.
    \item Jeśli $k$ przekracza 30, ale nie 2000,
    program dostanie 50\% punktów za test.
    \item Jeśli $k$ przekracza 20, ale nie przekracza $30$, to program otrzyma
    $50 + (30 - k) \cdot 5$ punktów.
    \item Jeśli powyższe warunki nie są spełnione, program dostanie 0
    \end{itemize}

\end{document}
