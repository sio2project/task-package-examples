\documentclass[zad,zawodnik,utf8,en]{sinol}
  \title{Guess the number}
  \id{gue}
  \contest{Example task}
  \Time{1}
  \RAM{64}

\begin{document}
\begin{tasktext}

    \noindent A two-person ``guess the number'' game is given.
    The rules are as follows:
    the first player chooses a number\\in the range [1, $n$],
    and gives the second player $n$ ($1 \leq n \leq 10^6$).
    The other player must guess the selected number. 
    He may ask questions saying ``is the secret number greater than $x$?''
    Write a program that will help the second player guess the number,
    so he doesn't have to use too many queries.

\section{Communication}

    Your program should use a library which allows querying the first player
    as well as reporting your final answer.
    To use the library, write the following in your program header:
    \begin{itemize}
        \item \textbf{C++:}
            \verb|#include "guelib.h"|

        \item \textbf{Python:}
            \texttt{from guelib import init, isGreater, answer}
    \end{itemize}


    \medskip\noindent
    The library provides the following three functions:
    \begin{itemize}
        \item $\texttt{init}()$ \\
            This function should be used once, as the first one, to begin the game.
            The function returns $n$ -- the upper bound of the range of numbers 
            which is used during this game.
            \begin{itemize}
                \item \textbf{C++:}
		                \texttt{int init();}

                \item \textbf{Python:}
		                \texttt{def init()}
            \end{itemize}

        \item $\texttt{isGreater}(x)$ \\
            Use this function to ask the first player,
            if number $x$ ($1 \leq x \leq n$)
            is greater than the number he picked.
            The function returns true if $x$ is greater, otherwise false.
            \begin{itemize}
                \item \textbf{C++:}
		                \texttt{bool isGreater(int x);}

                \item \textbf{Python:}
		                \texttt{def isGreater(x)}
            \end{itemize}

        \item $\texttt{answer}(x)$ \\
            This function should be called only once,
            at the termination of your program.
            It reports to the first player that $x$
            is your guess of his chosen number.
            \begin{itemize}
                \item \textbf{C++:}
		                \texttt{void answer(int x);}
                \item \textbf{Python:}
		                \texttt{def answer(x)}
            \end{itemize}
    \end{itemize}

    \noindent Your program \textbf{cannot} read any data,
    neither from standard input nor any files.
    Likewise, it \textbf{cannot} write to any files nor the standard output.
    It can write to the standard diagnostic output (\texttt{stderr}) -- remember
    though that this takes (precious) time.\\
    \\
    Your program should terminate immediately
    after calling the function \texttt{answer}.


\section{Sample execution of a program}

    \begin{tabular}{|c|c|p{12.5cm}|}
        \hline

        \textbf{Call} & \textbf{Return value} & \textbf{Explanation} \\ \hline

        \texttt{init()} & 5 &
        The game begins. The first player chooses number 4,
        and we look for a number in range [1, 5],
        which we are informed about by the value of the init() function.
        \\ \hline

        \texttt{isGreater(4)} & false &
        We ask if the number we are looking for is greater than 4.
        It is not greater (because it is equal),
        so we get the answer \texttt{false}.  
        \\ \hline

        \texttt{isGreater(3)} & true &
        We ask if the number we're looking for is greater than 3.
        It's greater, so we get the answer \texttt{true}.  
        \\ \hline

        \texttt{answer(4)} & - &
        We already know that the number we are looking for is 4,
        so we answer and finish the program.
        \\ \hline
    \end{tabular}



    \section{Grading}

    In order for the program to get any points, it must work as required,
    i.e. follow the rules described in the communication section.
    If it violates any of these rules, it will receive the ``Wrong Answer'' verdict.
    If the above conditions are met, the program is evaluated as follows.
    If the guessed number is incorrect, it will receive no points for the test.
    Otherwise, let $k$ be the number of queries.
    \begin{itemize}
        \item If $k$ does not exceed 20, the program will get
            the full number of points.
        \item If $k$ is over 30, but does not exceed 2000, the program will get
            50\% of points for the test.
        \item If $k$ is more than 20, but not more than 30,
            the program will receive $50 + (30 - k) \cdot 5$ points.
        \item If none of these conditions are met, the program will get 0 points.
    \end{itemize}


\section{Experiments}
    The directory \texttt{dlazaw} contains the following files,
    which allow testing if your program is formally correct:
    \begin{itemize}
        \item \textbf{C++:} a header file \texttt{guelib.h},
            a library \texttt{guelib.cpp} and an incorrect sample program
            \texttt{gue.cpp}
        \item \textbf{Python:} a library \texttt{guelib.py} and
            an incorrect sample program \texttt{gue.py}
    \end{itemize}

    \noindent Note that these libraries are merely for the sake of checking
    correctness of interaction and differ from those which will eventually
    evaluate your solution.
    While in the \texttt{dlazaw} directory,
    you can compile and run \texttt{gue.cpp} or \texttt{gue.py}.
    To compile in C++, type:

    \begin{itemize}
        \item \texttt{g++ -O3 -static gue.cpp guelib.cpp -std=c++11 -o gue}
    \end{itemize}

    \noindent Running the Python code doesn't need any extra steps to be taken.\\
    \\
    The resulting program reads from input the number $n$
    and in the next line the secret number.
    \\
    Remember that the provided sample library does not check
    if the input data is in the correct format,
    nor whether aforementioned conditions are satisfied.

\end{tasktext}
\end{document}
