\documentclass[zad,zawodnik,utf8]{sinol}
  \title{Liście}
  \id{lea}
  \contest{Zadanie przykładowe}
  \Time{1}
  \RAM{64}

\begin{document}
\begin{tasktext}

    \noindent Drzewo jest to spójny, nieskierowany i nieważony graf
    składający się z $n$ wierzchołków i $n-1$ krawędzi.
    Charakterystyczne dla drzewa jest to, że jest ono spójne i acykliczne.
    Liściem w drzewie jest wierzchołek, który ma stopień 1
    (czyli wychodzi z niego tylko jedna krawędź).
    Jeśli dane jest ukorzenione drzewo, to znaczy,
    że pewien wyznaczony wierzchołek jest korzeniem.
    Przyjmuje się, że w ukorzenionym drzewie korzeń nie jest nazywany liściem
    (nawet, jeśli jego stopień to 1).
    Dane jest $n$-wierzchołkowe ukorzenione drzewo.
    Korzeniem jest wierzchołek o indeksie 1.  
    Twoim zadaniem jest obliczyć liczbę jego liści.

\section{Wejście}

    W pierwszym wierszu wejścia podana jest jedna liczba $n$ ($n \leq 2 \cdot 10^5$)
    oznaczająca liczbę wierzchołków drzewa.
    Następne $n-1$ wierszy zawiera informacje o krawędziach w drzewie.
    W $i$-tym wierszu znajdują się dwie różne liczby $u_i$ i $v_i$
    ($1 \leq u_i, v_i \leq n$), oznaczające końce $i$-tej krawędzi.

\section{Wyjście}

    Wypisz jedną liczbę - liczbę liści w drzewie.

\examplesection
    \makecompactexample{0}

\section{Ocenianie}

    \begin{center}
        \begin{tabular}{|c|p{5cm}|c|}
            \hline
            \textbf{Podzadanie} & \textbf{Ograniczenia} & \textbf{Punkty} \\
            \hline
            1 & $n \leq 100$ & 30 \\
            \hline
            2 & Z korzenia wychodzą przynajmniej 2 krawędzie & 20 \\
            \hline
            3 & brak dodatkowych założeń & 50 \\
            \hline
        \end{tabular}
    \end{center}

\end{tasktext}
\end{document}
