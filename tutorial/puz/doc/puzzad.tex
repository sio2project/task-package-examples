\documentclass{spiral}
  \def\title{Puzzle}
  \def\id{puz}
  \def\contest{Example task}
  \def\desc{}
  \def\TL{1 s}
  \def\ML{256 MB}

\begin{document}
  \makeheader

  \noindent
  After some time John figured out a solution to a long forgotten puzzle.
  He wants to check if his idea is correct, so he asked you to provide a program for verification.
  The riddle goes something like this:\\
  ``There is given an area of $3 \times n$ size. Can you cover all of it with bricks of $1 \times 2$ size?''
  John thinks he knows when it's possible, but have some special requirements about your program.\\ \\
  The left lower corner of the area will be our $(0, 0)$ point.
  If the task is possible your program should give the integer $k$
  denoting the number of bricks needed for coverage.
  After that it should print exactly $k$ lines, each containing 3 integers $x$, $y$, $z$.
  The first two numbers denotes coordinate $(x, y)$ where the brick should be put.
  Last one defines, should it be put vertically or horizontally.
  If $z = 0$ the brick will be put horizontally (so it will cover fields $(x, y)$ and $(x + 1, y)$).
  If $z = 1$ the brick will be put vertically (it will cover fields $(x, y)$ and $(x, y + 1)$).
  Don't worry! If you know how many bricks John should put on the area,
  but you can't show him how to place them you can still get a reward!
  As you can see, there may exist more then one valid coverage. You can print \textbf{any valid one}.
  If the task is not possible, your program should print one line saying ``Can't do that''.


  \section{Input}

    In the first and only line of input there is
    a single integer $n$ ($1 \leq n \leq 100\,000$) denoting the length of the rectangle.

  \section{Output}

    If coverage is possible, the first line of standard output should contain exactly one integer $k$.
    Each of next $k$ lines should contain exactly 3 integers $x$, $y$, $z$.
    If $z = 0$ the brick will cover fields $(x, y)$ and $(x + 1, y)$.
    If $z = 1$ the brick will cover fields $(x, y)$ and $(x, y + 1)$.
    Remember that bricks \textbf{cannot} intersect!\\
    If coverage is not possible, print only one line saying ``Can't do that''.
    

  \example{0a}
  \example{0b}

\end{document}
