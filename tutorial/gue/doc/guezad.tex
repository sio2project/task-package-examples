\documentclass{spiral}
  \def\title{Zgadnij liczbę}
  \def\id{gue}
  \def\contest{Zadanie przykładowe}
  \def\desc{}
  \def\TL{1 s}
  \def\ML{64 MB}

\begin{document}
  \makeheader

  Dana jest dwuosobowa gra ,,zgadnij liczbę''. Zasady są następujące:
  pierwszy gracz wybiera liczbę z przedziału [1, $n$]
  oraz podaje drugiemu graczowi $n$ ($1 \leq n \leq 10^6$).
  Drugi gracz musi zgadnąć wybraną liczbę. 
  Do dyspozycji ma zapytania postaci ,,czy szukana liczba jest większa niż $x$?''
  Napisz program, który pomoże drugiemu graczowi zgadnąć liczbę tak,
  by nie musiał użyć zbyt wielu zapytań.

  \section{Komunikacja}
    Jest to zadanie interaktywne,
    to znaczy Twój program powinien używać biblioteki,
    która pozwala na zadawanie pytań pierwszemu graczowi.
    Aby użyć biblioteki, należy wpisać na początku programu:
    \begin{itemize}
        \item \textbf{C++:}
            \verb|#include "guelib.h"|

        \item \textbf{Python:}
            \texttt{from guelib import init, isGreater, answer}
    \end{itemize}
    Biblioteka udostępnia następujące trzy funkcje:
    \begin{itemize}
        \item $\texttt{init}()$ \\
            Funkcja ta powinna zostać wywoływana
            tylko raz, na początku działania programu.
            Rozpoczyna ona grę, a gracz pierwszy wymyśla szukaną liczbę.
            Zwraca liczbę $n$, podaną przez tego gracza.
            \begin{itemize}
                \item \textbf{C++:}
		                \texttt{int init();}

                \item \textbf{Python:}
		                \texttt{def init()}
            \end{itemize}

        \item $\texttt{isGreater}(x)$ \\
            Funkcja ta pyta pierwszego gracza,
            czy wybrana przez niego liczba jest większa od $x$.
            Zwraca \texttt{true} jeśli jest, \texttt{false} w przeciwnym wypadku.
            Podana liczba powinna należeć do przedziału [1, $n$].
            \begin{itemize}
                \item \textbf{C++:}
		                \texttt{bool isGreater(int x);}

                \item \textbf{Python:}
		                \texttt{def isGreater(x)}
            \end{itemize}

        \item $\texttt{answer}(x)$ \\
            Jest to funkcja, którą należy wywołać dokładnie raz,
            na koniec działania Twojego programu.
            Służy do podania odpowiedzi pierwszemu graczowi,
            a liczba $x$ jest zgadywaną przez nas wartością.
            \begin{itemize}
                \item \textbf{C++:}
		                \texttt{void answer(int x);}
                \item \textbf{Python:}
		                \texttt{def answer(x)}
            \end{itemize}
    \end{itemize}

    Twój program \textbf{nie może} czytać żadnych danych
    (ani ze standardowego wejścia, ani z plików).
    \textbf{Nie może} również nic wypisywać do plików ani na standardowe wyjście.
    Może pisać na standardowe wyjście diagnostyczne
    (\texttt{stderr}) -- pamiętaj jednak, że zużywa to cenny czas.

    Po użyciu funkcji \texttt{answer} Twój program powinien
    natychmiastowo zakończyć działanie.

\section{Przykładowe wykonanie programu}
    \begin{tabular}{|c|c|p{12.5cm}|}
        \hline

        \textbf{Funkcja} & \textbf{Wynik} & \textbf{Opis} \\ \hline

        \texttt{init()} & 5 &
        Gra się rozpoczyna. Gracz pierwszy wybiera liczbę 4,
        a szukamy liczby w przedziale [1, 5], o czym jesteśmy poinformowani
        wartością funkcji init().
        \\ \hline

        \texttt{isGreater(4)} & false &
        Pytamy czy szukana liczba jest większa od 4.
        Nie jest ona większa (gdyż jest ona równa),
        więc otrzymujemy odpowiedź \texttt{false}. 
        \\ \hline

        \texttt{isGreater(3)} & true &
        Pytamy czy szukana liczba jest większa od 3.
        Jest ona większa, więc otrzymujemy odpowiedź \texttt{true}. 
        \\ \hline

        \texttt{answer(4)} & - &
        Wiemy już, że szukaną liczbą jest 4,
        więc udzielamy odpowiedzi i kończymy program.
        \\ \hline
    \end{tabular}

\section{Ocenianie}
    Żeby program dostał jakiekolwiek punkty,
    musi on stosować się do zasad opisanych w sekcji komunikacji.
    Jeżeli naruszy on którąś z tych zasad, otrzyma werdykt ,,Błędna odpowiedź''.\\
    Jeśli powyższe warunki zostaną spełnione,
    to program oceniany jest w następujący sposób.\\
    Jeżeli zgadnięta liczba jest niepoprawna,
    nie otrzyma on punktów za dany test.\\
    W przeciwnym wypadku niech $k$ oznacza liczbę zapytań.
    \begin{itemize}
        \item Jeśli $k$ nie przekracza 20, program dostanie pełną liczbę punktów.
        \item Jeśli $k$ przekracza 30, ale nie 2000,
            program dostanie 50\% punktów za test.
        \item Jeśli $k$ przekracza 20, ale nie przekracza $30$, to program otrzyma
            $50 + (30 - k) \cdot 5$ punktów.
        \item Jeśli powyższe warunki nie są spełnione, program dostanie 0 punktów.
    \end{itemize}

\section{Eksperymenty}
    W katalogu \texttt{dlazaw} znajdują się pliki,
    które pozwolą Ci przetestować poprawność formalną rozwiązania.
    Możesz tam znaleźć następujące pliki:
    \begin{itemize}
        \item \textbf{C++:} plik nagłówkowy \texttt{guelib.h},
            bibliotekę \texttt{guelib.cpp}
            oraz przykładowy błędny program \texttt{gue.cpp}
        \item \textbf{Python:} bibliotekę \texttt{guelib.py}
            i przykładowy błędny program \texttt{gue.py}
    \end{itemize}

    Zwróć uwagę, że biblioteki te różnią się od tych,
    przy pomocy których będzie finalnie oceniane Twoje
    rozwiązanie, i służą jedynie do sprawdzenia poprawnej interakcji.
    Znajdując się w katalogu \texttt{dlazaw}, możesz
    standardowo skompilować i uruchomić \texttt{cyk.cpp} lub \texttt{cyk.py}.
    Przykładowa kompilacja w języku C++ to:

    \begin{itemize}
        \item \texttt{g++ -O3 -static gue.cpp guelib.cpp -std=c++11 -o gue}
    \end{itemize}

    Programy w Pythonie nie wymagają dodatkowych kroków do uruchomienia.

    Tak otrzymany program wczytuje z wejścia liczbę $n$,
    a w kolejnym wierszu sekretną liczbę $x$.

    Pamiętaj, że dostępna przykładowa biblioteka nie sprawdza,
    czy dane na wejściu są sformatowane poprawnie
    ani czy spełnione są ograniczenia z treści zadania.

\end{document}
